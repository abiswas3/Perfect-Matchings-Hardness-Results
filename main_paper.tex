\documentclass[11pt]{article}
\usepackage[margin=1in]{geometry} % Adjust margins as needed
%!TEX root = ./main_source.texx
\UseRawInputEncoding
\usepackage{hyperref}
% For authors with multiple institutions
\usepackage[affil-sl]{authblk}
% For line-numbers in my document
\usepackage{lineno}

% Redefine the cite color to your preferred color

\usepackage[utf8]{inputenc}
\usepackage[T1]{fontenc}
\usepackage{anyfontsize}
\usepackage{lipsum} % For dummy text
\usepackage{microtype} % Improved font rendering
\usepackage{titlesec} % Customize section titles
\usepackage{graphicx} % For including images
\usepackage{bm}
\usepackage{svg}
\usepackage{xcolor}
% \usepackage{ulem}
\usepackage{bbm}
\usepackage{xspace}
\usepackage{tabularx}
\usepackage{stmaryrd}
\SetSymbolFont{stmry}{bold}{U}{stmry}{m}{n}
\usepackage{bbm}
\usepackage{url}            % simple URL typesetting
\usepackage{booktabs}       % professional-quality tables
\usepackage{amsfonts}       % blackboard math symbols
\usepackage{nicefrac}       % compact symbols for 1/2, etc.
\usepackage{microtype}      % microtypography
\usepackage{amsmath,amsthm}
\usepackage{tikz}
\usepackage[most]{tcolorbox}

\usepackage{natbib}
% \bibliographystyle{dinat}
% \setcitestyle{authoryear, open={[},close={]}} %Citation-related commands

%\usepackage{amssymb}
\let\Bbbk\relax
\usepackage{newtxmath}
\usepackage{letltxmacro}
\usepackage{bm}

\usepackage{soul}
\usepackage[linesnumbered,ruled,vlined]{algorithm2e}
\usepackage[compatible]{algpseudocode} 
\usepackage{bbm}
\usepackage{svg}
\usepackage{xcolor}
\usepackage{xspace}
\usepackage{stmaryrd}
\usepackage{bbm}
\usepackage{url}            % simple URL typesetting
\usepackage{booktabs}       % professional-quality tables
\usepackage{nicefrac}       % compact symbols for 1/2, etc.
\usepackage{microtype}      % microtypography
\usepackage{letltxmacro}
\usepackage{bm}
\usepackage{caption}
\usepackage{subcaption}
\usepackage{soul}
\usepackage[linesnumbered,ruled,vlined]{algorithm2e}
\usepackage[compatible]{algpseudocode} 
% \usepackage{pgfplots}
\usepackage{doi}
\usepackage{fancyvrb,cprotect}
\usepackage{centernot}
\usepackage{fancyhdr}
\graphicspath{ {assets/} }


\definecolor{amber}{rgb}{1.0, 0.49, 0.0}
\definecolor{cadmiumgreen}{rgb}{0.0, 0.42, 0.24}
\definecolor{darkcyan}{rgb}{0.0, 0.55, 0.55}
\definecolor{darkcoral}{rgb}{0.8, 0.36, 0.27}
\definecolor{azure}{rgb}{0.0, 0.5, 1.0}
\definecolor{bittersweet}{rgb}{1.0, 0.44, 0.37}
\definecolor{razzmatazz}{rgb}{0.89, 0.15, 0.42}
\definecolor{ballblue}{rgb}{0.13, 0.67, 0.8}
\definecolor{purple}{rgb}{0.2, 0.2, 0.6}
\definecolor{egyptianblue}{rgb}{0.06, 0.2, 0.65}
\definecolor{darkslategray}{rgb}{0.0, 0.29, 0.29}
\definecolor{bananayellow}{rgb}{1.0, 0.88, 0.21}
\definecolor{blue-violet}{rgb}{0.54, 0.17, 0.89}
\definecolor{carminepink}{HTML}{EF58A0}
\definecolor{titlecolor}{RGB}{0,74,147}
\definecolor{sectioncolor}{RGB}{0,129,255}
\definecolor{subsectioncolor}{RGB}{0,129,255}
\definecolor{definitioncolor}{rgb}{0.89, 0.0, 0.13}
\usepackage{hyperref} 

\hypersetup{
    colorlinks=true,
    linkcolor=cadmiumgreen,
    citecolor=carminepink, % Set color for citation links
    urlcolor=titlecolor % Set color for URL links
}

\newcommand{\highlight}[1]{\color{razzmatazz} #1 \color{black}}
\newcommand{\ari}[1]{\textcolor{purple}{\textbf{[Ari]:} #1}}
\newcommand{\attentionC}[1]{\textcolor{orange}{\textbf{Attention Cl\'ement:}}\textcolor{blue}{\quad#1}}
\newcommand{\ykResolved}[1]{\textcolor{orange}{\textbf[YK] \sout{#1}}}
\newcommand{\CC}[1]{\textcolor{carminepink}{\textbf[CC] #1}}
\newcommand{\CCResolved}[1]{\textcolor{carminepink}{\textbf[CC] \sout{#1}}}
\usepackage{cleveref}       % smart cross-referencing


\usepackage{framed}
\usepackage{rotating}

\usepackage[colorinlistoftodos]{todonotes}
% todo leftbar
% \newenvironment{todo}{ %
% \def\FrameCommand{\hspace{-2em}%
% \begin{sideways}%
% \textcolor{red}{\textsf{\small TODO}}%
% \end{sideways}%
% \hspace{0.5em}\textcolor{red}{\vrule width 0.5pt} \hspace{0.5em}}\MakeFramed {\advance\hsize-\width \FrameRestore}}
% {\endMakeFramed}



%\newcommand{\email}[1]{\texttt{#1}}
%%%%%%%%%%%%%%%%%%%%%%%%%%%%%%
% Theorem
% Define a new theorem style
\newtheoremstyle{theoremstyle}
  {\topsep} % Space above
  {\topsep} % Space below
  {} % Body font
  {} % Indent amount
  {\bfseries\color{black}} % Theorem head font
  {} % Punctuation after theorem head
  {.5em} % Space after theorem head
  {\thmname{#1}~\thmnumber{#2}:~\textcolor{carminepink}{\thmnote{[#3]}}} % Theorem head spec (can be left empty, meaning ‘normal’)

% Define mdframed settings for the lemma box
\mdfdefinestyle{theoremframe}{
  linecolor=theoremcolor!80,
  linewidth=2pt,
  backgroundcolor=theoremcolor!15,
  roundcorner=5pt,
  nobreak=true,
  topline=false,
  bottomline=false,
  rightline=false,
}
% Apply the custom theorem style
\theoremstyle{theoremstyle}
% Define a theorem environment
\newtheorem{theorem}{Theorem}[section]
% Automatically surround lemma with mdframed
\surroundwithmdframed[style=theoremframe]{theorem}


%%%%%%%%%%%%%%%%%%%%%%%%%%%%%%
% Remark
\newtheoremstyle{remarkstyle}
  {\topsep} % Space above
  {\topsep} % Space below
  {} % Body font
  {} % Indent amount
  {\bfseries\color{black}} % Theorem head font
  {} % Punctuation after theorem head
  {0.5em} % Space after theorem head
  {} % Theorem head spec (can be left empty, meaning ‘normal’)

% Define mdframed settings for the lemma box
\mdfdefinestyle{remarkframe}{
  linecolor=fuyu-gaki!80,
  linewidth=2pt,
  backgroundcolor=fuyu-gaki!15,
  roundcorner=5pt,
  nobreak=true,
  topline=false,
  bottomline=false,
  rightline=false,
}
% Apply the custom theorem style
\theoremstyle{remarkstyle}
\newtheorem{remark}{Remark}
\surroundwithmdframed[style=remarkframe]{remark}
\newtheorem{corollary}{Corollary}
\surroundwithmdframed[style=remarkframe]{corollary}

\newtheoremstyle{definitionstyle}
  {\topsep} % Space above
  {\topsep} % Space below
  {} % Body font
  {} % Indent amount
  {\bfseries\color{black}} % Theorem head font
  {} % Punctuation after theorem head
  {.5em} % Space after theorem head
  {\thmname{#1}~\thmnumber{#2}:~\textcolor{carminepink}{\thmnote{[#3]}}} % Theorem head spec (can be left empty, meaning ‘normal’)

% Define mdframed settings for the lemma box
\mdfdefinestyle{definitionframe}{
  linecolor=definitioncolor!80,
  linewidth=2pt,
  backgroundcolor=definitioncolor!45,
  roundcorner=5pt,
  nobreak=true,
  topline=false,
  bottomline=false,
  rightline=false,
}
\theoremstyle{definitionstyle}
\newtheorem{definition}[theorem]{Definition}
\surroundwithmdframed[style=definitionframe]{definition}

%%%%%%%%%%%%%%%%%%%%%%%%%%%%%%
% Lemma
% Define a new theorem style
\newtheoremstyle{lemmastyle}
  {\topsep} % Space above
  {\topsep} % Space below
  {} % Body font
  {} % Indent amount
  {\bfseries\color{black}} % Theorem head font
  {} % Punctuation after theorem head
  {.5em} % Space after theorem head
  {\thmname{#1}~\thmnumber{#2}:~\textcolor{carminepink}{\thmnote{[#3]}}} % Theorem head spec (can be left empty, meaning ‘normal’)

% Define mdframed settings for the lemma box
\mdfdefinestyle{lemmaframe}{
  linecolor=theoremcolor!80,
  linewidth=2pt,
  backgroundcolor=theoremcolor!05,
  roundcorner=5pt,
  nobreak=true,
  topline=false,
  bottomline=false,
  rightline=false,
}
% Apply the custom theorem style
\theoremstyle{lemmastyle}
\newtheorem{lemma}[theorem]{Lemma}
% Automatically surround lemma with mdframed
\surroundwithmdframed[style=lemmaframe]{lemma}
\newtheorem{claim}[theorem]{Claim}
% Automatically surround lemma with mdframed
\surroundwithmdframed[style=definitionframe]{claim}

% Define custom environments for Question and Open Problem
% \newenvironment{question}
%   {\reversemarginpar
%    \marginnote{\textbf{\textcolor{fuyu-gaki}{Question:}}}}
%  {}

\newtcolorbox[auto counter, number within=section]{problem}[2][]{colframe=fuyu-gaki, colback=mizu!10, coltitle=white, title={Problem }, sharp corners, boxrule=0.8mm, width=0.99\textwidth, boxsep=2mm, left=3mm, right=3mm, top=2mm, bottom=2mm, breakable}



%%%%%%%%%%%%%%%%%%%%%%%%%%%%%%
\newcommand{\goal}[1]{
\begin{tcolorbox}[colframe=red!50!black, colback=white!95!black,title=Goal]
#1
\end{tcolorbox}    
}


% Common stuff that show up in nearly all writeups
\newcommand{\Def}{=}
\makeatletter
\newcommand{\smalldollar}{\mathrel{\mathpalette\small@dollar\relax}}
\newcommand{\small@dollar}[2]{%
  \vcenter{\hbox{%
    $#1\textnormal{\fontsize{0.7\dimexpr\f@size pt}{0}\selectfont\$}$%
  }}%
}
\makeatother
\renewcommand{\emph}[1]{\textit{#1}}
\newcommand\Bigger[2][7]{\left#2\rule{0mm}{#1truemm}\right.}
\renewcommand{\vec}[1]{\mathbf{#1}}
\newcommand{\TODO}{\textcolor{orange}{??TODO??}}
\renewcommand{\epsilon}{\varepsilon}
\newcommand{\Set}[1]{\left\{ #1\right\}}
\newcommand{\RCloseLOpenInterval}[2]{}
\newcommand{\LCloseROpenInterval}[2]{\left[#1, #2\right)}
\newcommand{\Interval}[2]{\left[\right]}
\newcommand{\OpenInterval}[2]{\left(\right)}
\newcommand{\DistSet}[1]{\Delta(#1)}
\newcommand{\Dist}{\mathcal{D}}
\newcommand{\leftarrowS}{\leftarrow\joinrel\smalldollar}
\newcommand{\rightarrowS}{\smalldollar\joinrel\rightarrow}
\newcommand{\samples}{\highlight{\leftarrowS}}
\newcommand{\negl}{\mathtt{negl}}
\newcommand{\Indicator}[1]{\mathbbm{1}\left\{#1\right\}}
\newcommand{\Eps}{\textcolor{red}{\varepsilon}}
\newcommand{\EpsLower}{\textcolor{shin-kai}{\varepsilon'}}
\newcommand{\EpsUpper}{\textcolor{shin-kai}{\varepsilon}}
\renewcommand{\tilde}[1]{\widetilde{#1}}
\newcommand{\bit}{\{0,1\}}
\newcommand{\poly}{\mathsf{poly}}
\newcommand{\Naturals}{\mathbb{N}}
\newcommand{\Integers}{\mathbb{Z}}
\newcommand{\Reals}{\mathbb{R}}
\newcommand{\Field}{\mathbb{F}}
\newcommand{\BigO}[1]{O\left(#1\right)}
\newcommand{\BigOTilde}[1]{\tilde{O}\left(#1\right)}
\newcommand{\SmallO}[1]{o\left(#1\right)}
\newcommand{\BigOmega}[1]{\Omega\left(#1\right)}
\newcommand{\SmallOmega}[1]{\omega\left(#1\right)}
\newcommand{\Mean}[2]{\mathbb{E}_{#1}\left[#2\right]}
\newcommand{\Prob}[1]{\Pr\left[#1 \right]}
\newcommand{\PProb}[2]{\Pr_{#2}\left[#1 \right]}
\newcommand{\True}{\texttt{True}}
\newcommand{\False}{\texttt{False}}


% Complexity classes 
\newcommand{\BPP}{\mathsf{BPP}}
\renewcommand{\P}{\mathsf{P}}
\newcommand{\NP}{\mathsf{NP}}
\newcommand{\CoNP}{\mathsf{coNP}}

% Paper specific + Proof Complexity Macros
\newcommand{\PropFormula}{\Phi}
\newcommand{\Proof}{\pi}
\newcommand{\Size}[1]{|#1|}
\newcommand{\Card}[1]{\text{Card}(#1)}
\newcommand{\Axioms}{\mathcal{Q}}
\newcommand{\axiom}{q}
\newcommand{\PM}[1]{\text{PM}(#1)}
\newcommand{\RemGraph}{G'}
\newcommand{\BadSet}{\textcolor{blue}{\hat{S}}}
\newcommand{\ari}[1]{\textcolor{kon-peki}{Ari: }#1}
\newcommand{\highlight}[1]{\textcolor{yama-budo}{#1}}
\newcommand{\red}[1]{\textcolor{fuyu-gaki}{#1}}
\newcommand{\green}[1]{\textcolor{shin-ryoku}{#1}}
\newcommand{\RedNodes}[1]{\red{\text{Red}}(#1)}
\newcommand{\GreenNodes}[1]{\green{\text{Green}}(#1)}
\newcommand{\GreenB}{\green{B}}
\newcommand{\RedA}{\red{A}}
\newcommand{\BipartiteG}{G_{\GreenB \cup \GreenB}}
\newcommand{\SizeRemGraph}{n'}

% Graph Commands
\newcommand{\Graph}{G}
\newcommand{\Vertices}[1]{\mathsf{V}(#1)}
\newcommand{\Edges}[1]{\mathsf{E}(#1)}
\newcommand{\CutEdges}[3]{e(#1 \leftrightarrow #2; #3)}
\newcommand{\IncidentEdges}[1]{\mathsf{E}(\rightarrow #1)}
\newcommand{\CutEdgesSet}[3]{\mathsf{E}(#1 \leftrightarrow #2; #3)}
\newcommand{\OddComponents}[1]{q(#1)}
\newcommand{\degree}[2]{\mathsf{deg}_{#1}(#2)} 
\newcommand{\MaxDegree}[1]{\Delta_{#1}}
\newcommand{\Neighbourhood}[2]{\Gamma_{#1}\left(#2\right)}
\newcommand{\Family}[1]{\textcolor{purple}{\mathcal{#1}}}
\newcommand{\SafeStates}{\Family{S}}
\newcommand{\HardInstance}{H}
\newcommand{\Embedding}[1]{\Psi\left(#1\right)}
\newcommand{\Path}[2]{\mathsf{P}_{#1 \rightsquigarrow #2
}}
\newcommand{\PathSize}{\textcolor{red}{l}}
\newcommand{\GlobalEdgeSet}{\textcolor{orange}{\mathcal{E}}}
\newcommand{\Degree}[1]{\text{Deg}\left(#1\right)}
\newcommand{\PerfectMatching}[1]{\mathsf{PM}\left(#1\right)}
\newcommand{\Complement}[1]{\overline{#1}}
\newcommand{\EdgeConnectivity}[1]{\kappa'(#1)}
\newcommand{\PC}{\vdash_{\texttt{PC}_F}}
\newcommand{\SOS}{\vdash_{\texttt{SOS}}}
\newcommand{\en}{\textcolor{ballblue}{n}}
\newcommand{\dee}{\textcolor{cadmiumgreen}{d}}
\newcommand{\eigen}{\textcolor{red}{\lambda}}
\newcommand{\EnDeeLambda}{(n, d, \lambda)}
\newcommand{\Subdivision}[2]{{#1}^{#2}}
\newcommand{\MinimalDegree}[1]{\delta_{#1}}
\newcommand{\dbtilde}[1]{\tilde{\raisebox{0pt}[0.85\height]{$\tilde{#1}$}}}
\newcommand{\na}{\green{n_B}}
\newcommand{\EdgesShort}{E}
\newcommand{\ExpansionFactor}[1]{\lambda(#1)}
\newcommand{\EmbeddingFunc}{\Psi}



 %These are generic commands (Might move them to %base template)is
\newcommand{\Degree}[1]{\mathsf{Deg}\left(#1\right)}
\newcommand{\PerfectMatching}[1]{\mathsf{PM}\left(#1\right)}


% \newcommand{\PC}{\underset{\texttt{PC}_F}{\vdash}}
\newcommand{\PC}{\vdash_{\texttt{PC}_F}}
\newcommand{\SOS}{\vdash_{\texttt{SOS}}}
\newcommand{\en}{\textcolor{ballblue}{n}}
\newcommand{\dee}{\textcolor{cadmiumgreen}{d}}
\newcommand{\eigen}{\textcolor{red}{\lambda}}
\newcommand{\EnDeeLambda}{(n, d, \lambda)}


\synctex=1
% Title
\title{\textcolor{definitioncolor}{Refuting Perfect Matchings In Expander Graphs Is Also Hard}}

% \author{}
\author[1]{Ari Biswas}
\author[2]{Rajko Nenadov}
\affil[1]{\small University Of Warwick, United Kingdom}
\affil[2]{\small University Of Auckland, New Zealand}

% ITCS guidelines.
% Authors should register and submit their paper via the hotcrp submission site, by the above strict deadlines. The font size should be at least 11 point and the paper should be single column. Beyond these, there are no formatting requirements. Authors are required to submit a COI declaration upon submission.
% Authors should strive to make their paper accessible not only to experts in their subarea, but also to the theory community at large. The submission should include clear proofs of all central claims. In addition, it is strongly recommended that the paper contain, within the first 10 pages, a concise and clear presentation of the merits of the paper, including a discussion of its significance, innovations, and place within (or outside) of our field's scope and literature. The committee will put a premium on writing that conveys clearly, in as simple and straightforward a manner as possible, what the paper accomplishes.
\date{}
% \linenumbers
\begin{document}

\maketitle
\begin{abstract}
Todo
\end{abstract}

\section{Introduction}


\begin{theorem}{Main Result}{thm:main-thm}
There is a constant $d_0 \in \Naturals$ such that for all $d \geq d_0$, the following holds asymptotically almost surely for any $(n, d, \lambda)$-graph $G$ on $n$ vertices, where $n$ is odd and $\lambda < ??$:
\begin{enumerate}
    \item{ $\Degree{\PerfectMatching{G} \PC \bot} = \BigOmega{\nicefrac{n}{\log n}}$} 
    \item{$\Degree{\PerfectMatching{G} \SOS \bot} = \BigOmega{\nicefrac{n}{\log n}}$}
    \item \highlight{add in the frege}
\end{enumerate}


\highlight{Figure out what the condition must be on $\lambda$}
\end{theorem}


\section{Technical Overview}

Put in proof by pictures here.

\begin{definition}
definition
\end{definition}

\section{Preliminaries}

\paragraph{Notation} Sets are denoted with upper case normal font e.g., $S$. Families (or collections) of sets are denoted with upper case calligraphic purple text e.g., $\Family{S}$. For a graph $\Graph$, we use $\Vertices{\Graph}$ and $\Edges{\Graph}$ to denote the vertices and edges of $\Graph$. 
Given $E' \subseteq \Edges{\Graph}$ and a vertex $v \in \Vertices{\Graph}$, we use $\degree{E'}{v}$ to denote the number of vertices adjacent to $v$ via edges in $E'$.

\subsection{Proof Complexity Basics}

\begin{definition}[Polynomial Calculus Refutations]\label{def:poly-calc-refutations}
	
\end{definition}

\subsection{Facts About Graphs}
\begin{definition}[$\EnDeeLambda$-Expander graphs]\label{def:expander-graphs}
Put in definition.    
\end{definition}

\begin{definition}[Topological minor]\label{def:top-minor}
Put in definition.    
\end{definition}


\begin{lemma}[Expander Mixing Lemma]
	
\end{lemma}


\section{Embedding Machinery}

The following definitions are adapted from \citep{nenadov2023routing} but were originally introduced by \citet{feldman1988wide} in their work about the theory of wide sense blocking networks.

\begin{definition}[$(p, q)$-non blocking bipartite graphs \citep{nenadov2023routing}]
Given $p,q \in \Naturals$ , we say that a bipartite graph $\Graph = (A \cup B, E)$ is $(p, q)$ \emph{non-blocking}, if there exists a family $\Family{S}$ of subsets of $E$, called the \emph{safe states}, such that the following holds:

\begin{enumerate}
	\item $\emptyset \in \Family{S}$
	\item If $E'' \subseteq E'$ and $E' \in \Family{S}$, \emph{then}, $E'' \in \Family{S}$.
	\item Given $E' \in \Family{S}$ such that $|E'| \leq p$, and \emph{any} $v \in A$ with $\degree{E'}{v} < q$, there exists $e = (v, w) \in E \setminus E'$  such that (i) $E' \cup \Set{e} \in \SafeStates $, and (ii) $w$ is not incident to any edge in $E'$ i.e. $\forall x \in A$ we have $(x,w) \notin E'$. 
\end{enumerate}

\end{definition}


\paragraph{Some geometric intuition about non-blocking graphs} \highlight{As long as the number of highways is not too large, i.e. less than $q$, I can always add a new destination to the network.}\par


Next, we re-state a lemma by \citep[Proposition 1]{feldman1988wide} which describes the properties bipartite graphs must satisfy to be non-blocking.

\begin{lemma}
Fix $a, r \in \Naturals$. Let $\Graph = (A \cup B, E)$ be a bipartite graph.
If for every $X \subseteq A$ of size $1 \leq |X| \leq 2a$, there are \emph{at least} $(r + s)|X|$ vertices in $B$ that are adjacent to $A$, \emph{then}, $\Graph$ is $(r, as)$-non blocking.
\end{lemma}

\paragraph{Construction:} Given a $\EnDeeLambda$-graph $\Graph$ on an odd number of vertices, define bipartite graph $\tilde{\Graph}  = A \cup B \Def \Vertices{\Graph} \cup \Vertices{\Graph}$.
That is, the left and right sides of the bipartite graph have the same vertices as that of the original graph $\Graph$.
For any $a \in A$ and $b \in B$, $(a,b) \in \Edges{\tilde{\Graph}}$ iff $(a,b) \in \Edges{\Graph}$.

\begin{lemma}[Show that my bipartite graph is non-blocking]\label{lemma: bipartite-construct-non-block}
The bipartite graph $\tilde{\Graph}$ constructed above is $(\pee, \qu)$ non-blocking.
\end{lemma}

\begin{proof}
\ari{I need to eventually express $\pee$ and $\qu$ as a function of $n, d, \lambda$.}
	
\end{proof}

\section{Proof Of Main Result}

\ari{Potentially use sub-division notation.}

Let $\Graph$ denote a $\EnDeeLambda$-graph on an odd number of vertices, and,  $\HardInstance$ denote the hard instance described by \highlight{Buss et al} that is worst case hard for polynomial calculus.
Let $\Vertices{\HardInstance} = \Set{h_1, \dots, h_k}$ be the vertices of the hard instance.
We break down the embedding process into the following steps:

\begin{enumerate}
	\item For each $j \in [k]$, our first task is to map $h_j$ to unique vertices $\Embedding{h_j} \in \Vertices{\Graph}$. 
	\item Secondly, for each edge $(u,v) \in \Edges{\HardInstance}$, we wish to find a path $\Path{u}{v}$ of odd length $\PathSize$.
	\item Successfully executing the above steps guarantees that $\HardInstance$ exists inside of $\Graph$ as a topological minor. But as described earlier, this does not suffice in showing that $\PerfectMatching{\Graph}$ is also hard to refute. Let $\Embedding{\HardInstance}$ denote the vertices that form the topological embedding of $\HardInstance$ in $\Graph$. We want to show that the induced subgraph $\Graph[\Vertices{\Graph} \setminus \Embedding{\HardInstance}]$ contains a perfect matching. 
	\item Let $M$ denote the matching described above. For all $e \in M$, we set $x_e =1$. For all $e \notin M$ that do not involve vertices in $\Embedding{\HardInstance}$, we set $x_e = 0$. Substituting these values in Definition \ref{def:poly-calc-refutations} gives rise to a restricted formula. Refuting the restricted formula reduces to refuting the hard instance of \highlight{Buss el al}. Thus, we are done.
\end{enumerate}

\subsection{Step 1}

Let $\tilde{\Graph} = A \cup B$ be the bipartite graph defined in the construction of Lemma \ref{lemma: bipartite-construct-non-block}.
We know that this graph $(\pee, \qu)$ is non-blocking.
Let $(a_1, \dots, a_k)$ denote an arbitrary ordering of arbitrary vertices in $A$.
Set $E' = \emptyset$. 
As $\tilde{\Graph}$ is non-blocking, there exists a collection of \emph{safe states} $\SafeStates$.


\subsection{Step 2}

\subsection{Step 3}

\section{Related Work}

\clearpage
\bibliographystyle{abbrvnat}
\bibliography{main_paper}
\end{document}
