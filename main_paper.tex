\documentclass[11pt]{article}
\usepackage[margin=1in]{geometry} % Adjust margins as needed
%!TEX root = ./main_source.texx
\UseRawInputEncoding
\usepackage{hyperref}
% For authors with multiple institutions
\usepackage[affil-sl]{authblk}
% For line-numbers in my document
\usepackage{lineno}

% Redefine the cite color to your preferred color

\usepackage[utf8]{inputenc}
\usepackage[T1]{fontenc}
\usepackage{anyfontsize}
\usepackage{lipsum} % For dummy text
\usepackage{microtype} % Improved font rendering
\usepackage{titlesec} % Customize section titles
\usepackage{graphicx} % For including images
\usepackage{svg}
\usepackage{xcolor}
\usepackage{ulem}
\usepackage{marginnote} % For margin notes

\usepackage{bbm}
\usepackage{xspace}
\usepackage{tabularx}
\usepackage{stmaryrd}
\SetSymbolFont{stmry}{bold}{U}{stmry}{m}{n}
\usepackage{bbm}
\usepackage{url}            % simple URL typesetting
\usepackage{booktabs}       % professional-quality tables
\usepackage{amsfonts}       % blackboard math symbols
\usepackage{nicefrac}       % compact symbols for 1/2, etc.
\usepackage{microtype}      % microtypography
\usepackage{amsmath,amsthm}
\usepackage{tikz}
\usepackage[most]{tcolorbox}
\usepackage{floatrow} % to get captions on the right
\usepackage{nicematrix}%<<<<<<<<<<<<<<
\usepackage[square,numbers]{natbib}

% \bibliographystyle{dinat}
% \setcitestyle{authoryear, open={[},close={]}} %Citation-related commands

%\usepackage{amssymb}
\let\Bbbk\relax
\usepackage{newtxmath}
\usepackage{letltxmacro}
\usepackage{bm}

\usepackage{soul}
\usepackage[linesnumbered,ruled,vlined]{algorithm2e}
\usepackage[compatible]{algpseudocode} 
\usepackage{bbm}
\usepackage{svg}
\usepackage{xcolor}
\usepackage{xspace}
\usepackage{stmaryrd}
\usepackage{bbm}
\usepackage{url}            % simple URL typesetting
\usepackage{booktabs}       % professional-quality tables
\usepackage{nicefrac}       % compact symbols for 1/2, etc.
\usepackage{microtype}      % microtypography
\usepackage{letltxmacro}
\usepackage{bm}
\usepackage[font=small,labelfont={bf},textfont={tt}]{caption} % Monospace captions
\usepackage{subcaption} % For subfigures
\usepackage{soul}
\usepackage[linesnumbered,ruled,vlined]{algorithm2e}
\usepackage[compatible]{algpseudocode} 
% \usepackage{pgfplots}
\usepackage{doi}
\usepackage{fancyvrb,cprotect}
\usepackage{centernot}
\usepackage{fancyhdr}
\graphicspath{ {assets/} }

\usepackage{hyperref} 

\hypersetup{
    colorlinks=true,
    linkcolor=ama-iro,
    citecolor=carminepink, % Set color for citation links
    urlcolor=titlecolor % Set color for URL links
}

\usepackage{cleveref}       % smart cross-referencing


\usepackage{framed}
\usepackage{rotating}

\usepackage[colorinlistoftodos]{todonotes}

% These are for theorems.
\UseRawInputEncoding
% \usepackage[conf={},createShortEnv]{proofs-end} 
\usepackage{thmtools}
\usepackage[framemethod=TikZ]{mdframed}

% Set margin width and alignment
\usepackage{marginnote} % For margin notes
\usepackage{xcolor}     % For colors
\usepackage{lipsum}     % For dummy text

% Set margin width and alignment
\renewcommand*{\raggedrightmarginnote}{\raggedright}
\setlength{\marginparwidth}{2.5cm}



\definecolor{amber}{rgb}{1.0, 0.49, 0.0}
\definecolor{cadmiumgreen}{rgb}{0.0, 0.42, 0.24}
\definecolor{darkcyan}{rgb}{0.0, 0.55, 0.55}
\definecolor{darkcoral}{rgb}{0.8, 0.36, 0.27}
\definecolor{azure}{rgb}{0.0, 0.5, 1.0}
\definecolor{bittersweet}{rgb}{1.0, 0.44, 0.37}
\definecolor{razzmatazz}{rgb}{0.89, 0.15, 0.42}
\definecolor{ballblue}{rgb}{0.13, 0.67, 0.8}
\definecolor{purple}{rgb}{0.2, 0.2, 0.6}
\definecolor{egyptianblue}{rgb}{0.06, 0.2, 0.65}
\definecolor{darkslategray}{rgb}{0.0, 0.29, 0.29}
\definecolor{bananayellow}{rgb}{1.0, 0.88, 0.21}
\definecolor{blue-violet}{rgb}{0.54, 0.17, 0.89}
\definecolor{carminepink}{HTML}{C71585} % Magenta-pink, use for "prove or disprove" statements
\definecolor{titlecolor}{RGB}{0,74,147}
\definecolor{sectioncolor}{RGB}{0,129,255}
\definecolor{subsectioncolor}{RGB}{0,129,255}
\definecolor{definitioncolor}{rgb}{251, 74, 52}
\definecolor{theoremcolor}{RGB}{48, 51, 50}

% iroshizuku color names
\definecolor{ama-iro}{RGB}{0, 158, 243.0} % Light blue, can be used for headers or highlights
\definecolor{fuyu-gaki}{HTML}{C75146} % Persimmon red, ideal for warnings or critical points
\definecolor{momiji}{RGB}{245, 70, 111} % Bright red, great for emphatic notes or alerts
\definecolor{hotaru-bi}{RGB}{229,221,58} % Yellowish, for tips or hints
\definecolor{kon-peki}{RGB}{1,120,217} % Deep blue, suitable for section titles
\definecolor{shin-kai}{RGB}{77,98,152} % Steel blue, good for subtle text or links
\definecolor{shin-ryoku}{RGB}{1,145,97} % Green, use for success messages or "OK" status
\definecolor{yama-budo}{RGB}{171,14,122} % Deep magenta, excellent for key points or notes
\definecolor{tsutsuji}{HTML}{C71585} % Magenta-pink, use for "prove or disprove" statements
\definecolor{mizu}{rgb}{0.0, 0.6, 0.8} % Water (light blue), good for background shading
\definecolor{kikyou}{rgb}{0.4, 0.4, 0.8} % Bellflower (blueish purple), appropriate for emphasis or subdued headers
\definecolor{theoremcolor}{RGB}{77,98,152}
\definecolor{definitioncolor}{RGB}{255, 213, 179}
\definecolor{examplecolor}{RGB}{255, 213, 179}


\usepackage{accents}
\usepackage{calc}

%%%%%%%%%%%%%%%%%%%%%%%%%%%%%%
% Theorem
% Define a new theorem style
\newtheoremstyle{theoremstyle}
  {\topsep} % Space above
  {\topsep} % Space below
  {} % Body font
  {} % Indent amount
  {\bfseries\color{black}} % Theorem head font
  {} % Punctuation after theorem head
  {.5em} % Space after theorem head
  {\thmname{#1}~\thmnumber{#2}:~\textcolor{theoremcolor}{\thmnote{[#3]}}} % Theorem head spec (can be left empty, meaning ‘normal’)

% Define mdframed settings for the lemma box
\mdfdefinestyle{theoremframe}{
  linecolor=theoremcolor!80,
  linewidth=2pt,
  backgroundcolor=theoremcolor!15,
  roundcorner=5pt,
  nobreak=true,
  topline=false,
  bottomline=false,
  rightline=false,
}
% Apply the custom theorem style
\theoremstyle{theoremstyle}
% Define a theorem environment
\newtheorem{theorem}{Theorem}[section]
% Automatically surround lemma with mdframed
\surroundwithmdframed[style=theoremframe]{theorem}


%%%%%%%%%%%%%%%%%%%%%%%%%%%%%%
% Remark
\newtheoremstyle{remarkstyle}
  {\topsep} % Space above
  {\topsep} % Space below
  {} % Body font
  {} % Indent amount
  {\bfseries\color{black}} % Theorem head font
  {} % Punctuation after theorem head
  {0.5em} % Space after theorem head
  {} % Theorem head spec (can be left empty, meaning ‘normal’)

% Define mdframed settings for the lemma box
\mdfdefinestyle{remarkframe}{
  linecolor=hotaru-bi!80,
  linewidth=2pt,
  backgroundcolor=hotaru-bi!15,
  roundcorner=5pt,
  nobreak=true,
  topline=false,
  bottomline=false,
  rightline=false,
}
% Apply the custom theorem style
\theoremstyle{remarkstyle}
\newtheorem{remark}{Remark}
\surroundwithmdframed[style=remarkframe]{remark}
\newtheorem{corollary}[theorem]{Corollary}
\surroundwithmdframed[style=remarkframe]{corollary}

% Definitions
\newtheoremstyle{definitionstyle}
  {\topsep} % Space above
  {\topsep} % Space below
  {} % Body font
  {} % Indent amount
  {\bfseries\color{black}} % Theorem head font
  {} % Punctuation after theorem head
  {.5em} % Space after theorem head
  {\thmname{#1}~\thmnumber{#2}:~\textcolor{theoremcolor}{\thmnote{[#3]}}} % Theorem head spec (can be left empty, meaning ‘normal’)

% Define mdframed settings for the lemma box
\mdfdefinestyle{definitionframe}{
  linecolor=definitioncolor!80,
  linewidth=2pt,
  backgroundcolor=definitioncolor!45,
  roundcorner=5pt,
  nobreak=true,
  topline=false,
  bottomline=false,
  rightline=false,
}
\theoremstyle{definitionstyle}
\newtheorem{definition}[theorem]{Definition}
\surroundwithmdframed[style=definitionframe]{definition}

%%%%%%%%%%%%%%%%%%%%%%%%%%%%%%
% Lemma
% Define a new theorem style
\newtheoremstyle{lemmastyle}
  {\topsep} % Space above
  {\topsep} % Space below
  {} % Body font
  {} % Indent amount
  {\bfseries\color{black}} % Theorem head font
  {} % Punctuation after theorem head
  {.5em} % Space after theorem head
  {\thmname{#1}~\thmnumber{#2}:~\textcolor{theoremcolor}{\thmnote{[#3]}}} % Theorem head spec (can be left empty, meaning ‘normal’)

% Define mdframed settings for the lemma box
\mdfdefinestyle{lemmaframe}{
  linecolor=theoremcolor!80,
  linewidth=2pt,
  backgroundcolor=theoremcolor!05,
  roundcorner=5pt,
  nobreak=true,
  topline=false,
  bottomline=false,
  rightline=false,
}
% Apply the custom theorem style
\theoremstyle{lemmastyle}
\newtheorem{lemma}[theorem]{Lemma}
% Automatically surround lemma with mdframed
\surroundwithmdframed[style=lemmaframe]{lemma}
\newtheorem{claim}[theorem]{Claim}
% Automatically surround lemma with mdframed
\surroundwithmdframed[style=definitionframe]{claim}

% Define custom environments for Question and Open Problem
% \newenvironment{question}
%   {\reversemarginpar
%    \marginnote{\textbf{\textcolor{fuyu-gaki}{Question:}}}}
%  {}

\newtcolorbox[auto counter, number within=section]{problem}[2][]{colframe=fuyu-gaki, colback=mizu!10, coltitle=white, title={Problem }, sharp corners, boxrule=0.8mm, width=0.99\textwidth, boxsep=2mm, left=3mm, right=3mm, top=2mm, bottom=2mm, breakable}



%%%%%%%%%%%%%%%%%%%%%%%%%%%%%%
\newcommand{\goal}[1]{
\begin{tcolorbox}[colframe=red!50!black, colback=white!95!black,title=Goal]
#1
\end{tcolorbox}    
}


% Common stuff that show up in nearly all writeups
\newcommand{\Def}{=}
\makeatletter
\newcommand{\smalldollar}{\mathrel{\mathpalette\small@dollar\relax}}
\newcommand{\small@dollar}[2]{%
  \vcenter{\hbox{%
    $#1\textnormal{\fontsize{0.7\dimexpr\f@size pt}{0}\selectfont\$}$%
  }}%
}
\makeatother
\renewcommand{\emph}[1]{\textit{#1}}
\newcommand\Bigger[2][7]{\left#2\rule{0mm}{#1truemm}\right.}
\renewcommand{\vec}[1]{\mathbf{#1}}
\newcommand{\TODO}{\textcolor{orange}{??TODO??}}
\renewcommand{\epsilon}{\varepsilon}
\newcommand{\Set}[1]{\left\{ #1\right\}}
\newcommand{\RCloseLOpenInterval}[2]{}
\newcommand{\LCloseROpenInterval}[2]{\left[#1, #2\right)}
\newcommand{\Interval}[2]{\left[\right]}
\newcommand{\OpenInterval}[2]{\left(\right)}
\newcommand{\DistSet}[1]{\Delta(#1)}
\newcommand{\Dist}{\mathcal{D}}
\newcommand{\leftarrowS}{\leftarrow\joinrel\smalldollar}
\newcommand{\rightarrowS}{\smalldollar\joinrel\rightarrow}
\newcommand{\samples}{\highlight{\leftarrowS}}
\newcommand{\negl}{\mathtt{negl}}
\newcommand{\Indicator}[1]{\mathbbm{1}\left\{#1\right\}}
\newcommand{\Eps}{\textcolor{red}{\varepsilon}}
\newcommand{\EpsLower}{\textcolor{shin-kai}{\varepsilon'}}
\newcommand{\EpsUpper}{\textcolor{shin-kai}{\varepsilon}}
\renewcommand{\tilde}[1]{\widetilde{#1}}
\newcommand{\bit}{\{0,1\}}
\newcommand{\poly}{\mathsf{poly}}
\newcommand{\Naturals}{\mathbb{N}}
\newcommand{\Integers}{\mathbb{Z}}
\newcommand{\Reals}{\mathbb{R}}
\newcommand{\Field}{\mathbb{F}}
\newcommand{\BigO}[1]{O\left(#1\right)}
\newcommand{\BigOTilde}[1]{\tilde{O}\left(#1\right)}
\newcommand{\SmallO}[1]{o\left(#1\right)}
\newcommand{\BigOmega}[1]{\Omega\left(#1\right)}
\newcommand{\SmallOmega}[1]{\omega\left(#1\right)}
\newcommand{\Mean}[2]{\mathbb{E}_{#1}\left[#2\right]}
\newcommand{\Prob}[1]{\Pr\left[#1 \right]}
\newcommand{\PProb}[2]{\Pr_{#2}\left[#1 \right]}
\newcommand{\True}{\texttt{True}}
\newcommand{\False}{\texttt{False}}


% Complexity classes 
\newcommand{\BPP}{\mathsf{BPP}}
\renewcommand{\P}{\mathsf{P}}
\newcommand{\NP}{\mathsf{NP}}
\newcommand{\CoNP}{\mathsf{coNP}}

% Paper specific + Proof Complexity Macros
\newcommand{\PropFormula}{\Phi}
\newcommand{\Proof}{\pi}
\newcommand{\Size}[1]{|#1|}
\newcommand{\Card}[1]{\text{Card}(#1)}
\newcommand{\Axioms}{\mathcal{Q}}
\newcommand{\axiom}{q}
\newcommand{\PM}[1]{\text{PM}(#1)}
\newcommand{\RemGraph}{G'}
\newcommand{\BadSet}{\textcolor{blue}{\hat{S}}}
\newcommand{\ari}[1]{\textcolor{kon-peki}{Ari: }#1}
\newcommand{\rajko}[1]{\textcolor{red}{Rajko: #1}}
\newcommand{\highlight}[1]{\textcolor{yama-budo}{#1}}
\newcommand{\red}[1]{\textcolor{fuyu-gaki}{#1}}
\newcommand{\green}[1]{\textcolor{shin-ryoku}{#1}}
\newcommand{\RedNodes}[1]{\red{\text{Red}}(#1)}
\newcommand{\GreenNodes}[1]{\green{\text{Green}}(#1)}
\newcommand{\GreenB}{\green{B}}
\newcommand{\RedA}{\red{A}}
\newcommand{\BipartiteG}{G_{\GreenB \cup \GreenB}}
\newcommand{\SizeRemGraph}{n'}

% Graph Commands
\newcommand{\Graph}{G}
\newcommand{\Vertices}[1]{\mathsf{V}(#1)}
\newcommand{\Edges}[1]{\mathsf{E}(#1)}
\newcommand{\CutEdges}[3]{e(#1 \leftrightarrow #2; #3)}
\newcommand{\IncidentEdges}[1]{\mathsf{E}(\rightarrow #1)}
\newcommand{\CutEdgesSet}[3]{\mathsf{E}(#1 \leftrightarrow #2; #3)}
\newcommand{\OddComponents}[1]{q(#1)}
\newcommand{\degree}[2]{\mathsf{deg}_{#1}(#2)} 
\newcommand{\MaxDegree}[1]{\Delta_{#1}}
\newcommand{\Neighbourhood}[2]{\Gamma_{#1}\left(#2\right)}
\newcommand{\Family}[1]{\textcolor{purple}{\mathcal{#1}}}
\newcommand{\SafeStates}{\Family{S}}
\newcommand{\HardInstance}{H}
\newcommand{\Embedding}[1]{\Psi\left(#1\right)}
\newcommand{\Path}[2]{\mathsf{P}_{#1 \rightsquigarrow #2
}}
\newcommand{\PathSize}{\textcolor{red}{l}}
\newcommand{\GlobalEdgeSet}{\textcolor{orange}{\mathcal{E}}}
\newcommand{\Degree}[1]{\text{Deg}\left(#1\right)}
\newcommand{\PerfectMatching}[1]{\mathsf{PM}\left(#1\right)}
\newcommand{\Complement}[1]{\overline{#1}}
\newcommand{\EdgeConnectivity}[1]{\kappa'(#1)}
\newcommand{\PC}{\vdash_{\texttt{PC}_F}}
\newcommand{\SOS}{\vdash_{\texttt{SOS}}}
\newcommand{\en}{\textcolor{ballblue}{n}}
\newcommand{\dee}{\textcolor{cadmiumgreen}{d}}
\newcommand{\eigen}{\textcolor{red}{\lambda}}
\newcommand{\EnDeeLambda}{(n, d, \lambda)}
\newcommand{\Subdivision}[2]{{#1}^{#2}}
\newcommand{\MinimalDegree}[1]{\delta_{#1}}
\newcommand{\dbtilde}[1]{\tilde{\raisebox{0pt}[0.85\height]{$\tilde{#1}$}}}
\newcommand{\na}{\green{n_B}}
\newcommand{\EdgesShort}{E}
\newcommand{\ExpansionFactor}[1]{\lambda(#1)}
\newcommand{\EmbeddingFunc}{\Psi}



 %These are generic commands (Might move them to %base template)is
\newcommand{\Degree}[1]{\mathsf{Deg}\left(#1\right)}
\newcommand{\PerfectMatching}[1]{\mathsf{PM}\left(#1\right)}


% \newcommand{\PC}{\underset{\texttt{PC}_F}{\vdash}}
\newcommand{\PC}{\vdash_{\texttt{PC}_F}}
\newcommand{\SOS}{\vdash_{\texttt{SOS}}}
\newcommand{\en}{\textcolor{ballblue}{n}}
\newcommand{\dee}{\textcolor{cadmiumgreen}{d}}
\newcommand{\eigen}{\textcolor{red}{\lambda}}
\newcommand{\EnDeeLambda}{(n, d, \lambda)}
\newcommand{\Subdivision}[2]{{#1}^{#2}}
\newcommand{\MinimalDegree}[1]{\delta(#1)}
%\newcommand{\Indicator}[1]{\mathbbm{1}\left\{#1\right\}}
\newcommand{\dbtilde}[1]{\tilde{\raisebox{0pt}[0.85\height]{$\tilde{#1}$}}}

\newcommand{\EdgesAccross}[2]{e_{#2}(#1)}

\newcommand{\DistSet}[1]{\Delta(#1)}
\newcommand{\Dist}{\mathcal{D}}
\newcommand{\EdgesShort}{E}
%\newcommand{\Mean}[2]{\underset{#1}{\mathbb{E}}\left[#2\right]}
\newcommand{\Mean}[2]{\mathbb{E}_{#1}\left[#2\right]}
\newcommand{\Prob}[1]{\Pr\left[#1 \right]}
\newcommand{\PProb}[2]{\Pr_{#2}\left[#1 \right]}
\newcommand{\ExpansionFactor}[1]{\lambda(#1)}
\synctex=1
% Title
\title{\textcolor{definitioncolor}{Refuting Perfect Matchings In Expander Graphs Is Also Hard}}
\newcommand{\dApproxLower}{\textcolor{red}{d'}}
\newcommand{\dApproxExpandedLower}{\textcolor{black}{\epsilon d - \sqrt{\nicefrac{d}{2}\log (2/\delta)}}}
\newcommand{\dApproxExpandedUpper}{\textcolor{black}{\epsilon d + \sqrt{\nicefrac{d}{2}\log (2/\delta)}}}


% \author{}
\author[1]{Ari Biswas}
\author[2]{Rajko Nenadov}
\affil[1]{\small University Of Warwick, United Kingdom}
\affil[2]{\small University Of Auckland, New Zealand}

% ITCS guidelines.
% Authors should register and submit their paper via the hotcrp submission site, by the above strict deadlines. The font size should be at least 11 point and the paper should be single column. Beyond these, there are no formatting requirements. Authors are required to submit a COI declaration upon submission.
% Authors should strive to make their paper accessible not only to experts in their subarea, but also to the theory community at large. The submission should include clear proofs of all central claims. In addition, it is strongly recommended that the paper contain, within the first 10 pages, a concise and clear presentation of the merits of the paper, including a discussion of its significance, innovations, and place within (or outside) of our field's scope and literature. The committee will put a premium on writing that conveys clearly, in as simple and straightforward a manner as possible, what the paper accomplishes.
\date{}
% \linenumbers
\begin{document}

\maketitle
\begin{abstract}
Todo
\end{abstract}

\section{Introduction}


\begin{theorem}{Main Result}{thm:main-thm}
There is a constant $d_0 \in \Naturals$ such that for all $d \geq d_0$, the following holds asymptotically almost surely for any $(n, d, \lambda)$-graph $G$ on $n$ vertices, where $n$ is odd and $\lambda < ??$:
\begin{enumerate}
    \item{ $\Degree{\PerfectMatching{G} \PC \bot} = \BigOmega{\nicefrac{n}{\log n}}$} 
    \item{$\Degree{\PerfectMatching{G} \SOS \bot} = \BigOmega{\nicefrac{n}{\log n}}$}
    \item \highlight{add in the frege}
\end{enumerate}


\highlight{Figure out what the condition must be on $\lambda$}
\end{theorem}


\section{Technical Overview}

Put in proof by pictures here.



\section{Preliminaries}

\paragraph{Notation} Sets are denoted with upper case normal font e.g., $S$. Families (or collections) of sets are denoted with upper case calligraphic purple text e.g., $\Family{S}$. 


\begin{definition}[Dependency Graphs]
	
\end{definition}

\begin{lemma}[Lov\`asz Local Lemma]\label{lemma:lll}Let $\Dist \in \DistSet{\bit^*}$ be a discrete probability distribution.
Let $E_1,...,E_n$ be a set of events, and assume that the following hold:

\begin{enumerate}
	\item $\PProb{E_i}{\Dist} \leq \delta$ for some $\delta \in (0,1)$.
	\item The degree of the dependency graph given by $E_1, \dots, E_n$ is bounded by $d$
	\item $d\delta \leq \nicefrac{1}{4}$
	
Then, 

\[ \PProb{\overset{n}{ \underset{i=1}{\cap}} \hspace{0.1cm}  \overline{E_i}}{\Dist} > 0\]	
\end{enumerate}
	
\end{lemma}

\subsection{Proof Complexity Preliminaries}

\begin{definition}[Polynomial Calculus Refutations]\label{def:poly-calc-refutations}
	
\end{definition}

\subsection{Graph Theory Preliminaries}

For a graph $\Graph$, we use $\Vertices{\Graph}$ and $\Edges{\Graph}$ to denote the vertices and edges of $\Graph$. 
For any $E' \subseteq \Edges{\Graph}$ and a vertex $v \in \Vertices{\Graph}$, we use $\Neighbourhood{E'}{v} = \{ u \in \Vertices{\Graph} : (u,v) \in E' \}$ to denote the neighbourhood of $v$ with respect to $E'$, and $\degree{E'}{v}$ to denote the number of vertices adjacent to $v$ via edges in $E'$.

\begin{definition}[Subgraph]\label{def:subgraph}
A graph $G'=(V', E')$ is a subgraph of another graph $G=(V, E)$ iff (1) $V'\subseteq V$, and (2) $E'\subseteq E$ and  $(v1, v2) \in E' \implies (v1, v2) \in V')$.
	
\end{definition}

\begin{definition}[Sub-divisions]\label{def:subdivisions}
Given a graph $H$ and a function $\sigma: \Edges{H} \rightarrow \Naturals$, the $\sigma$-subdivision $H$, denoted by $\Subdivision{H}{\sigma}$, is the graph obtained by replacing each edge in $\Edges{H}$ with a path of length $\sigma(e)$ joining the end points of $e$ (such that all these paths are mutually vertex disjoint, except at the end points).	
\end{definition}

\begin{definition}[Topological Minor]\label{def:topological-minor}
A graph $H$ is a topological minor of a graph $G$ if there exists a subdivision $\Subdivision{H}{\sigma}$ that is isomorphic to \emph{any} subgraph of $G$.	
\end{definition}

Next we define pseudorandom graphs or expander graphs. 
Throughout this document, it suffices to treat $d$ as a small constant.

\begin{definition}[$\EnDeeLambda$ pseudorandom graphs]\label{def:expander-graphs}
Let $G$ be a $d$-regular graph on $n$ vertices, and, let $\lambda_1 \geq \lambda_2, \dots, \geq \lambda_n$ denote eigenvalues of the adjacency matrix of $G$.
We say $G$ is an $\EnDeeLambda$-graph if $\ExpansionFactor{\Graph} \Def \underset{{\{2, \dots, n\}}}{\max}|\lambda_i| \leq \lambda$.
\end{definition}


\begin{lemma}[Expander Mixing Lemma]\label{lemma:expanders-mixing-lemma}
	
\end{lemma}


\section{Embedding Machinery}

The following definitions are adapted from \citep{nenadov2023routing} but were originally introduced by \citet{feldman1988wide} in their work about the theory of wide sense blocking networks.

\begin{definition}[$(p, q)$-non blocking bipartite graphs \citep{nenadov2023routing}]
Given $p,q \in \Naturals$ , we say that a bipartite graph $\Graph = (A \cup B, E)$ is $(p, q)$ \emph{non-blocking}, if there exists a family $\Family{S}$ of subsets of $E$, called the \emph{safe states}, such that the following holds:

\begin{enumerate}
	\item $\emptyset \in \Family{S}$
	\item If $E'' \subseteq E'$ and $E' \in \Family{S}$, \emph{then}, $E'' \in \Family{S}$.
	\item Given $E' \in \Family{S}$ such that $|E'| \leq p$, and \emph{any} $v \in A$ with $\degree{E'}{v} < q$, there exists $e = (v, w) \in E \setminus E'$  such that (i) $E' \cup \Set{e} \in \SafeStates $, and (ii) $w$ is not incident to any edge in $E'$ i.e. $\forall x \in A$ we have $(x,w) \notin E'$. 
\end{enumerate}

\end{definition}


\paragraph{Why are non-blocking graphs useful} \highlight{Use the get out of jail analysis I wrote down.}\par
\begin{figure}[h]
\center
%\floatbox[{\capbeside\thisfloatsetup{capbesideposition={right,top},capbesidewidth=4cm}}]{figure}[\FBwidth]
{\caption{A test figure with its caption side by side}\label{fig:test}}
{\includegraphics{assets/non-blocking-networks.pdf}}
\end{figure}

Next, we re-state a lemma by \citep[Proposition 1]{feldman1988wide} which describes the properties bipartite graphs must satisfy to be non-blocking.

\begin{lemma}\label{lemma:condtions-for-non-block}
Fix $a, r, s \in \Naturals$. Let $\Graph = (A \cup B, E)$ be a bipartite graph.
If for every $X \subseteq A$ of size $1 \leq |X| \leq 2a$, there are \emph{at least} $(r + s)|X|$ vertices in $B$ that are adjacent to $A$, \emph{then}, $\Graph$ is $(r, as)$-non blocking.
\end{lemma}


\section{Proof Of Main Result}

Let $\Graph$ denote a $\EnDeeLambda$-graph on an odd number of vertices, and,  $\HardInstance$ denote the hard instance, as described by \highlight{Buss et al} that is worst case hard for the corresponding proof system.


\begin{enumerate}
	\item First, we show (via Lemma \ref{lemma:partition}) that we can partition the vertices of the graph $\Graph$ into two disjoint sets $A$ and $B$, as shown in Figure \ref{fig:partition}. The partition has the property that for each vertex $v \in V$, has roughly $\epsilon d$ neighbours inside $A$ i.e. $|\Neighbourhood{E}{v} \cap A| \approx \epsilon d$, where $\epsilon$ is a small enough constant like 0.1.
	\item Secondly, we find a subdivision $\Subdivision{H}{\sigma}$ of $H$, where for each $e \in \Edges{H}$, we have $\sigma(e) = \PathSize =2k+1$ for some $k = \lceil 1/2\log_{D}n\rceil$ (Note that $\PathSize$ is odd by definition). Next we show (via Lemma \ref{lemma:top-embedding}) that $\Subdivision{H}{\sigma}$ is a subgraph of $\Graph[A]$.
	
%	\item Successfully executing the above steps guarantees that $\HardInstance$ exists inside of $\Graph$ as a topological minor. But as described earlier, this does not suffice in showing that $\PerfectMatching{\Graph}$ is also hard to refute. Let $\Embedding{\HardInstance}$ denote the vertices that form the topological embedding of $\HardInstance$ in $\Graph$. We want to show that the induced subgraph $\Graph[\Vertices{\Graph} \setminus \Embedding{\HardInstance}]$ contains a perfect matching. 
	\item We show (via Lemma \ref{lemma:perfect-matching}) that $\Graph[B]$ has a perfect matching $M$. For all $e \in M$, we set $x_e = 1$ and for all $e \notin \Edges{\Subdivision{H}{\sigma}}$ but not in $M$, set $x_e = 0$.
\end{enumerate}

\subsection{Step 1}

\begin{figure}
	\includegraphics{assets/non-blocking-networks.pdf}
	\caption{Partition \highlight{Change this}}
	\label{fig:partition}
\end{figure}

Figure \ref{fig:partition} pictorially describes Lemma \ref{lemma:partition}.

\begin{lemma}\label{lemma:partition}Given a $\EnDeeLambda$ graph $\Graph = (V, E)$, there exists a small constant $\epsilon \in (0,1)$ and a partition of $V$ into disjoint sets $A$ and $B$, such that, we have for every $v \in V$, $|\Neighbourhood{\EdgesShort}{a} \cap A| \leq \epsilon d + \sqrt{\nicefrac{d}{2}\log (2/\delta)}$,
where $\delta \in (0,1)$ such that $\delta d \leq 1/4$.	
\end{lemma}

\begin{proof}We prove the existence of such a partition $A \cup B = V$ using the probabilistic method.
For each $v \in \Vertices{\Graph}$, we toss an independent coin $X_i$ with bias $\epsilon$.
If $X_i = 1$, then we include $v$ in $A$, else we put $v$ in $B$.
Thus, $\vec{X} \Def (X_1, \dots, X_n) \in \bit^n$ is a random variable that describes how we partition $\Vertices{G}$ into sets $A$ and $B$.
For any $v \in V$, let $Y_v = |\Neighbourhood{\EdgesShort}{v} \cap A| = \sum_{e \in \Neighbourhood{E}{v}} \Indicator{e \in A}$ denote the random variable that counts how many neighbours of $a$ are in $A$.
%If $a \in B$, then $Y_a = 0$.
%Fix $\delta \in (0,1)$ small enough such that $\delta d \leq 1/4$.
For any $v \in V$, let $E_v = \Indicator{|Y_v -\epsilon d| \geq \sqrt{d/2\log 2/\delta}}$ denote the bad event that $Y_v$ is very far from $\epsilon d$.
As $\Graph$ is $d$-regular, $\Mean{\vec{X}}{Y_v} = \epsilon d$, and that the dependency graph of events $\{ E_v \}_{v \in V}$ has max-degree at most $d$.
By Chernoff-Hoeffding, for any $v \in V$, we have $\PProb{E_v}{\vec{X}} \leq \delta$.
Finally, by the assumption, $\delta d \leq 1/4$, we can 
invoke the Lov\`asz Local Lemma (Lemma \ref{lemma:lll}) to get $\PProb{\forall a \in A, \text{ } E_a = 0}{\vec{X}} > 0$.
\end{proof}

%Let $\tilde{\Graph} = A \cup B$ be the bipartite graph defined in the construction of Lemma \ref{lemma: bipartite-construct-non-block}.
%We know that this graph $(\pee, \qu)$ is non-blocking.
%Let $(a_1, \dots, a_k)$ denote an arbitrary ordering of arbitrary vertices in $A$.
%Set $E' = \emptyset$. 
%As $\tilde{\Graph}$ is non-blocking, there exists a collection of \emph{safe states} $\SafeStates$.


\subsection{Step 2}

Let $G=(V,E)$, be the $\EnDeeLambda$ graph given to us. 
Let $\epsilon, \delta, A$ and $B$ be as defined by Lemma \ref{lemma:partition}. 
Define $\dApproxLower \Def \dApproxExpandedLower$.
Define $E_A \Def \{ (u,v) \in E : u \in A \land v \in A\}$. 
That is $E_A$ are the edges whose endpoints are fully contained in A, and let $V_A \Def \{ v \in A: \text{$v$ is an endpoint of some $e \in E_A$}\}$ and let $ n_A \Def |V_A|$.
Now define a bipartite graph $G_A = (V_A \cup V_A)$ such that if $(u,v) \in E_A$, then $(u,v) \in \Edges{G_A}$.

\begin{lemma}\label{lemma:bipartitie-is-non-blocking}Let $G$ be a $\EnDeeLambda$ graph. Set $D = \nicefrac{d}{\lambda}$. If $\lambda \leq \nicefrac{d}{C_1}$, then construction defined above is $(D, \alpha n)$ non-blocking for some $\alpha \in (0,1)$.
\end{lemma}
\begin{proof}
Let $X$ and $Y$ denote the left and right vertices of the bipartite graph $G_A$.
Let $a = \frac{\lambda n}{c_1d}$. Consider a subset $S \subseteq X$ of size $1 \leq |S| \leq 2a \leq \frac{\lambda n}{c_1'd}$. 
Suppose towards contradiction of Lemma \ref{lemma:condtions-for-non-block}, there exists $T \subseteq Y$ of vertices adjacent to $S$ of size $|T| \leq 2(d/\lambda)|X|$.

From Lemma \ref{lemma:partition}, we know that for any $v \in V_A$, $\Neighbourhood{E_A}{v} \geq \dApproxLower$.	
Therefore, for any $S\subseteq X$ and $T \subseteq Y$, we have 

\begin{equation}
\EdgesAccross{S, T}{E_A}	\geq |S| \dApproxLower
\end{equation}

As $G$ is an $\EnDeeLambda$ graph, we have that $G_A$ is a $(n_A, \dApproxLower, \lambda)$ graph.
By the expander mixing lemma (Lemma \ref{lemma:expanders-mixing-lemma}, 

\[ \EdgesAccross{S, T}{E_A}	\leq \frac{\dApproxLower}{n_A}|S||T| + \lambda\sqrt{|S||T|}\]

\end{proof}

\begin{lemma}[Topological Embedding]\label{lemma:top-embedding}	
\end{lemma}

\begin{proof}
For each $h \in \Vertices{H}$, our first task is to define $\Embedding{h} \in \Vertices{G}$.

Le
	
\end{proof}

\subsection{Step 3}

Lemma \ref{lemma:expanders-minimal-matching} is from \citep[Theorem 4.3]{krivelevich2006pseudo}, that says the pseudorandom graphs on an even number of vertices have perfect matchings.

\begin{lemma}\label{lemma:expanders-minimal-matching}
Let $\Graph$ be an $\EnDeeLambda$ graph with $\lambda \leq d - 2$	and $n$ is even. Then $\Graph$ has a perfect matching.
\end{lemma}

\begin{lemma}\label{lemma:perfect-matching}
Given $\EnDeeLambda$ graph $\Graph=(V,E)$, let $\Embedding{\Subdivision{H}{\sigma}}$ be the topological embedding of the hard instance $H$ from Lemma \ref{lemma:top-embedding}.
If $\lambda \leq \nicefrac{d}{C}$ for some large enough constant $C > 0$, then the sub-graph $\Graph[V \setminus \Embedding{\Subdivision{H}{\sigma}}]$ has a perfect matching.	
\end{lemma}

\begin{proof}
Let $\epsilon, \delta, A$ and $B$ be as defined in Lemma \ref{lemma:partition}.
Define $\tilde{d} = 1 - (\epsilon d + \sqrt{\nicefrac{d}{2}\log (2/\delta)})$
Let $\Embedding{\Subdivision{H}{\sigma}}$ denote the subgraph of $G$ which represents the topological embedding of $H$ from Lemma \ref{lemma:top-embedding}.
Define $\tilde{G} = \Graph[V \setminus \Embedding{\Subdivision{H}{\sigma}}]$. 
That is $\tilde{\Graph}$ is a subgraph of $\Graph$ using all vertices in $B$ and the vertices in $A$ that were not used to embed $\Subdivision{H}{\sigma}$.
As $\ExpansionFactor{\Graph} \leq \lambda$, and $\tilde{\Graph}$ is a subgraph, we have $\ExpansionFactor{\tilde{\Graph}} \leq \lambda$.
%\ari{This is super easy to show using basic linear algebra, I can explicitly prove this if you think it's needed.}
By lemma \ref{lemma:partition}, we know that the minimal degree $\MinimalDegree{\tilde{\Graph}} \geq \tilde{d}$.
Let $\dbtilde{\Graph}$ denote a subgraph of $\tilde{G}$, where for each $v \in \Vertices{\tilde{\Graph}}$ we retain exactly $\tilde{d}$ of its neighbours.
It is easy to see that $\dbtilde{\Graph}$ is a $(\tilde{n}, \tilde{d}, \lambda)$ pseudorandom graph, where $\tilde{n} = |\Vertices{\tilde{\Graph}}|$ and $n$ is even (since $|\Vertices{\Graph}|$ is odd and $|\Vertices{\Embedding{\Subdivision{H}{\sigma}}}|$ is also odd). 
As $\lambda \leq \tilde{d} - 2$, we invoke Lemma \ref{lemma:expanders-minimal-matching} to get that $\dbtilde{G}$ has a perfect matching.
As $\dbtilde{G}$ is a sub-graph of $\tilde{G}$ with the same vertex set, $\tilde{G}$ also has a perfect matching.
\ari{Set $C$ here so that this is less complex}.
\end{proof}

\section{Related Work}

\clearpage
\bibliographystyle{abbrvnat}
\bibliography{main_paper}
\end{document}
